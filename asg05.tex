\documentclass {article}
\usepackage{amsmath}
\setlength{\parindent}{0cm}

\usepackage{hyperref}
\usepackage{graphicx}
\usepackage{subcaption}
\usepackage[margin=1.5in]{geometry}

\usepackage{fancyhdr}
\pagestyle{fancy}
\lhead{\textbf{Complex Network Analysis} \\ Assignment 5\\}
\rhead{Maria Kagkeli \\ Maria Regina Lily \\ Mihai Verzan}
\headheight 10pc
\voffset -10pc

\begin{document}


\section*{Problem 5-1: Role of Preferential Attachment}

\subsection*{1}
\begin{align*}
 \frac{dk_{i}}{dt} \approx 
 m \Pi k_{i} \approx 
 m \cdot \frac{1}{m_0 + k-1} = 
 \frac{m}{m_0 + t -1}&&   \text{(5.12)}
\end{align*}

\subsection*{3}
\begin{align*}
  m &\left[ 1 + \log  \frac{m_0 + t - 1}{m_0 + 1 - m_0 + (m_0 + t - 1)\exp \left( 1 - \frac{k}{m} \right)}  \right]  \\ \\
  = m &\left[ 1 + \log( m_0 + t - 1) - \log (m_0 + t - 1) \exp \left( 1 - \frac{k}{m}\right) \right] \\ \\
  = m &\left[ 1 + \log(m_0 + t - 1) - (\log m_0 + t - 1)  - \log \left( \exp \left( 1 - \frac{k}{m} \right) \right) \right] \\ \\
  = m &\left[ 1 - 1 + \frac{k}{m} \right] \\ \\
  = k   
\end{align*}

\subsection*{5}
\begin{align*}
 P(k) & = \frac{dP(k)}{dk} = \frac{e^{1 - \frac{k}{m}}}{m} &&&&
\end{align*}

\newpage
\section*{Problem 5-2: Friendship Paradox}
\subsection*{1}
\subsection*{4}
People with lots of friends are more likely to be counted as one of your friends in the first place.

\newpage
\section*{Problem 5-3: Barabási-Albert Model}

For source code, see \texttt{asg005.ipynb}

\subsection*{2}

\begin{tabular}{l | c c }
    & Practical Results & Analytical Results \\
\hline
Number of Nodes& 105 & 105 \\
Number of Edges& 310 & 310\\
Sum of Node Degree& 620 &  -
\end{tabular}
\medskip

Based on these values, we can confirm that our \texttt{barabasi\_albert} function is correct

\subsection*{3}

\begin{tabular}{l | c c }
    & Practical Results & Analytical Results \\
\hline
Average Clustering Coefficient& 0.0512 & 0.0475 \\
Diameter                      & 5 & 3.575\\
$\gamma$                      & 10.9&  3

\end{tabular}
\medskip


%TO DO: comment on results (avg clus coeff and gamma).
We obtained a higher value for the diameter of our generated network than the expected value according the lecture slides. However it is not too far off, and the formula presented in the lecture is for the average diameter of all possible Barabási-Albert networks based on their size, so it is OK to deviate slightly from this average.

\end{document}
